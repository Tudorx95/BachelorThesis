\documentclass[../../main.tex]{subfiles}


\begin{document}

\chapter{Introducere}

\section{Context}

Odata cu dezvoltarea sistemelor de calcul moderne si a componentelor Hardware, s-au putut realiza produse software complexe cu capacitati de stocare
net superioare. Revolutia tehnologica a permis nu doar realizarea unor sarcini simple, precum calcule matematice, sau automatizarea unor dispozitive ( de exemplu aprinderea automata a unui bec printr-un microcontroler),
ci si posibilitatea gestionarii mai eficiente a informatiilor digitale (de la date bancare la fisiere media).

Aceasta a devenit treptat principala sursa legitima de inregistrare a oricarui tip de date (text, imagini, video, audio).
Pentru a accesa si actualiza informatia digitala, s-au dezvoltat diferite versiuni de baze de date centralizate si distribuite.

Bazele de date centralizate sunt aplicatii software specializate ce folosesc resursele sistemului (a statiei) pentru a raspunde cat mai rapid interogarilor.
Statiile trebuie sa detina multa putere de stocare si de procesare in comparatie cu un sistem de calcul normal destinat utilizatorilor casnici.
In alta ordine de idei, s-au dezvoltat si baze de date distribuite, menite sa reduca din capacitatile tehnice ale serverului si sa stocheze informatia sub
forma descentralizata. Cautarea resursei in acest context ar presupune interogarea recursiva a fiecarei entitati pana la gasirea sa. Prin acest mod, nu doar
ca statiile pot avea si capabilitati tehnice mai reduse, dar si pot pastra copii de rezerva (backup) locale pentru fiecare segment de informatie in parte.

Această evoluție naturală spre descentralizare a deschis drumul unor concepte moderne precum învățarea automată federată (federated learning), unde datele nu mai sunt transferate către un server central. În schimb, modelele sunt antrenate local, iar parametrii sunt ulterior agregați global. Astfel, se menține confidențialitatea datelor, fără a compromite performanța modelului.

Evolutia tehnologica continua a dat nastere la o serie de atacuri cibernetice menite sa destabilizeze securitatea aplicatiilor si totodata sustragerea a
cat mai multe date sau identitati private in contradictie cu normele legale.
Cele mai populare atacuri raportate la scara globala pentru anul curent 2025 sunt ransomware (conform \footnotemark{}, in SUA s-au raportat cresteri
de 149\%), furtul de identitate prin exfiltrarea de credentiale, si phishing.
Dezvoltarea modelelor de inteligență artificială a amplificat aceste riscuri, oferind atacatorilor instrumente automatizate pentru generarea și adaptarea atacurilor.

Pentru a limita utilizarea abuzivă a tehnologiilor bazate pe AI, Uniunea Europeană a adoptat în 2024 un set de reglementări stricte privind integrarea acestor module în aplicațiile software, prin AI Act \footnotemark{}.

Progresul din domeniul machine learning si a Large Language Models a fost posibil ca urmare a unui volum masiv de date disponibile si a nevoii tot mai mari de analiza.
Acest lucru a determinat aparitia unei noi categorii de specialisti, data scientists, dedicati colectarii si prelucrarii minutioase a datelor pentru antrenarea modelelor.

Totuși, pe măsură ce investițiile în tehnologii AI au crescut, au apărut și actori rău intenționați care încearcă să exploateze vulnerabilitățile din procesul de antrenare.
Întrucât modelele moderne depind de calitatea datelor folosite, acestea au devenit o țintă principală a atacurilor.
Atacatorii se regasesc si ei intr-o pozitie constanta de adaptare la noile formalitati de securitate si incearca sa contracareze fiecare element nou.
Astfel, avand in vedere complexitatea dezvoltarii unui modul de inteligenta artificiala specializat pe diferite domenii, tinta s-a redirectionat spre volumul de date pe care acestea le folosesc si care pot determina starea finala a aplicatiei.

În contextul învățării automate distribuite, literatura de specialitate identifică trei categorii majore de atacuri:

- Atacuri asupra datelor, precum data poisoning, unde setul de antrenare este manipulat pentru a altera comportamentul modelului;

- Atacuri asupra modelului, prin modificarea parametrilor sau a gradientului (de exemplu, model poisoning);

- Atacuri asupra canalului de comunicare, care vizează interceptarea sau modificarea mesajelor dintre entitățile participante.

Lucrarea de față se concentrează pe prima categorie, data poisoning, în cadrul unei infrastructuri de invatare federate.

\footnotetext[1]{https://www.dnsc.ro/vezi/document/buletin-de-indicatori-statistici-si-tendinte-de-securitate-cibernetica-h1-2025}
\footnotetext[2]{ https://artificialintelligenceact.eu/wp-content/uploads/2024/11/Future-of-Life-InstituteAI-Act-overview-30-May-2024.pdf}

\section{Motivatia lucrarii}

Avand in vedere aspectele legate de posibilitatea unei interventii asupra setului de date de antrenare, atac denumit otravire a datelor (data poisoning),
munca cercetatorilor s-a ingreunat. Preocuparea nu mai este primordial asupra analizei setului de date de antrenare, cat despre mentinerea integritatii si a confidentialitatii lor.
Pentru a răspunde acestor nevoi, colaborarea dintre cercetători s-a orientat către modele distribuite de lucru, iar învățarea federată (federated learning) a devenit una dintre principalele direcții.
Aceasta permite colaborarea între participanți fără a partaja direct seturile lor de date, menținând o barieră naturală împotriva accesului neautorizat. Totuși, deși infrastructura este diferită față de abordările centralizate, vulnerabilitățile rămân, iar atacurile asupra datelor utilizate local pot afecta modelul global.

In urma unei analize proprii, am putut observa diferite solutii/frameworks de simulare a procesului de invatare automata federata, dar fara o integrare cu mecanisme moderne de testare pentru atacuri
precum otravirea datelor (data poisoning) amintite anterior \footnotemark{}. Unele dintre aceste framework-uri sunt poate dificil de gestionat si configurat \footnotemark{}, si nu permit extinderea usoara prin integrare altor componente.
In acelasi timp, gandindu-ne la multitudinea de atacuri malware si la platformele de detectie a lor, devine clar ca in domeniul inteligentei artificiale lipseste o platforma centralizata, flexibila, dedicata testarii si evaluarii cu diferite tipuri de atacuri asupra modelelor distribuite.

Aceste limitări justifica realizarea prezentei lucrari, care își propune dezvoltarea unei platforme de simulare capabile sa testeze atacuri de tip data poisoning într-o infrastructura de învățare federată.

\footnotetext[3]{ https://ibmfl-api-docs.res.ibm.com/index.html }
\footnotetext[4]{ https://github.com/IBM/federated-learning-lib/tree/main }

\section{Obiectivele lucrarii}

Plecand de la neajunsurile prezentate, ne propunem in aceasta lucrare sa venim in sprijinul comunitatii de cercetare stiintifica in domeniul securizarii solutiilor cu AI cu o platforma de simulare cu sursa deschisa ("open source"), a acestei clase de atacuri pe mai multe directii.
Astfel, oferim cercetatorilor posibilitatea analizei algoritmului de antrenare propriu dezvoltat, plecand de la o retea neuronala de baza si un set de date uzual (imagini), si testarea sa prin antrenare in diferite conditii.
Platforma in sine respecta toate normele unei aplicatii software de productie, in care fiecare actiune are propria sa logica de implementare.
Serviciile sunt segregate suficient de mult incat sa permita o dezvoltare ulterioara prin integrarea lor cu alte sisteme.

Rezultatele pot fi utile in contextul securizarii procesului de antrenare al algoritmului, dar si pentru analiza factorilor de risc la care e expus in acest mediu.

Cercetatorul este cel care furnizeaza algoritmul python de antrenare a propriei retele neuronale sau algoritm de Machine Learning. El seteaza parametrii simularii atat pentru procesul de antrenare, cat si pentru tipul de atac de otravire a datelor.
Platforma isi propune sa simuleze acest tip de atac cu ajutorul acestor setari de inceput intr-un mediu de invatare federata, furnizand la final o comparatie intre modelul antrenat folosind datele normale de antrenare si cel antrenat cu datele otravite.
Aceste rezultate pot fi utile in semnalarea unui posibil risc la nivelul modelului dezvoltat, oferind mai apoi solutii de imbunatatire a implementarii sale.


\section{Structura lucrarii}

% Aici completam la final 


% Structura pe capitole 
\chapter{Notiuni Teoretice}

In acest capitol, vor fi prezentate notiunile teoretice specifice intelegerii procesului de dezvoltare a platformei de simulare.
Vom incepe cu Notiunile introductive despre conceptele de Machine Learning in antiteza cu Deep Learning. In continuare, vom discuta despre invatarea federata si arhitectura unei infrastructuri federate de invatare automata, tipurile
de atacuri data poisoning implementate in procesul de simulare a atacurilor, precum si alte notiuni specifice implementarii.

    \section{Notiuni introductive}

    Machine Learning si Deep Learning sunt doua ramuri importante ale Inteligentei Artificiale care au rolul dezvoltarii unor modele specifice rezolvarii unor anumite
    actiuni. Pornind de la antrenarea de retele neuronale, ne orientam atentia spre setul de date de antrenare si spre actorii ce pot interveni in acest proces.
    Mediul in care testam ofera o perspectiva reala asupra impactului pe care il pot avea aceste atacuri la nivelul unei organizatii sau aplicatii.

    \subsection{Diferenta dintre Machine Learning si Deep Learning}

    Inteligența Artificială (AI) este domeniul vast care înglobează orice tehnică ce permite calculatoarelor să imite comportamentul uman.
Informatia a evoluat treptat odata cu imbunatatirea capabilitatilor de stocare ale dispozitivelor si aparitia programelor software complexe.
De la simplul format de text, inregistrari audio, pana la imagini si video in rezolutii 4K, modul de lucru s-a diversificat constant.

La fel au evoluat si cerintele utilizatorilor, care tind sa acceada catre solutii automate care sa le rezolve problemele uzuale,
precum identificarea de patterns in imagini sau chiar din video, sau generarea de text.

IA vine sa rezolve aceste probleme si sa introduca algoritmi de rezolvare specifici pentru fiecare tip de informatie furnizata.

Machine Learning este o componenta importanta din domeniul IA care se diferentiaza de alte metode de antrenare prin optimizarile pe care le aduce erorilor ce apar din predictia rezultatului corect.
Modelele de ML clasice se bazeaza pe interventia umana in factorul de decizie (supervised learning), mai precis datele de intrare sunt etichetate pentru a oferi un context de predictie stabil.

Deep Learning este o subcategorie a Machine Learning, care are rolul de a minimiza interventia umana si a automatiza procesul de decizie. Prin aceasta metoda se automatizeaza mare parte din extragerea caracteristicilor pe setul de date, eliminand nevoia de a defini etichete pentru fiecare valoare de intrare (unsupervied learning).

Diferenta dintre aceste doua concepte este in modul in care acesti algoritmi invata si procentul de utilizare a datelor [1].
Scopul principal al invatarii automate este predictia. Pe baza unui set de date de antrenare si de testare, se determina o anumita categorie de iesire predefinita.

    \subsection{Retea Neuronala}

    Retelele Neuronale sunt un subset al Machine Learning si se identifica drept infrastructura de baza din cadrul algoritmilor de Deep Learning. Denumirea de "neuronal" se refera la structura lor interna, in care fiecare caracteristica (feature) este un neuron ce interactioneaza unii cu altii.
Ele sunt compuse din 3 straturi/layers: primul strat il reprezinta stratul nodurilor de intrare, al doilea strat este denumit "stratul ascuns" (hidden layer) pt ca incapsuleaza mai multe straturi, iar ultimul strat este cel de iesire in care se face predictia propriu-zisa.
Straturile ascunse sunt concepute pentru a procesa iterativ datele pornind de la starea lor din nodurile de intrare pana la stratul de iesire.


    \section{Invatare automata federata}

    Evolutia hardware in tehnologie a condus la cresterea numarului dispozitivelor mobile (telefoane, tablete), denumite gadgets din faptul ca sunt mici, portabile si moderne.
Ele au fost mai departe adoptate la scara larga, devenind obiecte indispensabile in era tehnologica ce avea sa vina.

    \subsection{Concept}

    Invatarea automata federata permite lucrul cu modele de ML sau chiar retele neuronale, antrenate distribuit, pe un numar mare de dispozitive in scopul rezolvarii unei probleme de IA. Distribuirea sarcinilor a fost adoptata si in contextul opozitiei lucrului centralizat, pe servere ce detin capabilitati Hardware performante (placi grafice de ultima generatie), dar care genereaza costuri mari si care pot fi predispuse la amenintari de securitate cibernetica, fiind considerate SPOF(Single Point of Failure).    

    \subsection{Arhitectura FL}


In literatura, exista mai multe categorii de arhitecturi de invatare automata federata. In aceasta sectiune ne vom concentra pe clasificarea generala a arhitecturii unei aplicatii folosind federated learning, si vom enumera pentru o anumita categorie cum se clasifica dispozitivele utilizate.

Federated Learning, asa cum a mai fost mentionat, este organizat dintr-un server (agregator) si multiple dispozitive client. Modul in care aceste entitati comunica este fundamentul principal in modului de imbunatatire al invatarii.

In modul clasic al federated learning, dispozitivele client transmit actualizari ale modelului de baza la un server central care aplica asupra lor o functie de agregare, reconstruind intreg modelul de baza. Aceasta setare/model, presupune de fapt o delegare a sarcinii de invatare, insa pastreaza entitate centrala necesara imbunatatirii solutiei.
Acest fapt, nu tine sa evita posibilitatea amenintarilor cibernetice (Single Point of Failure), ci doar sa usureze costurile centralizatorului in a procesa local problema, distribuind sarcinile.

Modelul Fully decentralized (peer-to-peer) learning, ofera o noua abordare si rezolva problema cibernetica amintita. In aceasta setare nu exista agregator, imbunatatirile fiind comunicate intre clienti interconectati. Ideea principala se bazeaza pe inlocuirea comunicarii cu agregatorul cu cea intre dispozitive individuale printr-un protocol prestabilit.
In functie de numarul de dispozitive, se concepe un graf de conexiuni in care fiecare nod reprezinta un client, iar fiecare muchie un canal de comunicatie. Restrictia principala este ca un dispozitiv sa fie conectat la un numar maxim limitat de dispozitive adiacente, prestabilit, in contradictie cu un graf complet (stea) specific arhitecturii clasice client-server.
Nodurile isi imbunatatesc propriile variante ale retelei, si isi comunica rezultatele pe care le agrega local, realizand o medie a ponderilor.
In comparatie cu modelul federated learning clasic, modelul fully decentralized nu specifica de la inceput dispozitivelor un model de baza global de la care sa porneasca in procesul de rezolvare a problemei.

\begin{figure}[h!]
    \centering
    \includegraphics[width=0.7\textwidth]{../images/FL_models.png}
    \caption{Imaginea 2.2.2: Modele Federated Learning}
    \label{fig:nume-label}
\end{figure}

Imaginea de mai sus ofera o privire de ansamblu asupra celor doua modele de arhitecturi si caracteristicile acestora.

\begin{landscape}
    \begin{table}[h]
        \centering
        \begin{tabular}{ | p{0.20\linewidth} | p{0.38\linewidth} | p{0.38\linewidth} | }
            \hline
            \textbf{Model} & \textbf{Avantaje} & \textbf{Dezavantaje} \\
            \hline
    
            \textbf{Federated learning} (centralized coordination)
            &
            \begin{itemize}
                \item mai simplu de configurat (topologie hub-and-spoke)
                \item pornește de la un model global de bază
                \item agregarea centralizată reduce sarcina de calcul pe clienți
                \item necesită mai puține conexiuni (doar client → server)
                \item gestiunea și monitorizarea sunt mai simple
            \end{itemize}
            &
            \begin{itemize}
                \item SPOF (Single Point of Failure) – serverul central
                \item serverul poate deveni țintă pentru atacuri
                \item nu elimină riscurile cibernetice, doar distribuie munca
                \item dependența de coordonator pentru progresul antrenării
                \item necesită infrastructură centralizată permanent disponibilă
            \end{itemize}
            \\
            \hline
    
            \textbf{Fully decentralized / peer-to-peer learning}
            &
            \begin{itemize}
                \item previne atacurile specifice unui server central (evită SPOF)
                \item reziliență crescută – compromiterea unui nod nu destabilizează întregul sistem
                \item îmbunătățirile se propagă prin graf, fără entitate centrală
                \item agregare locală (fiecare nod mediază ponderile)
                \item poate scala natural dacă graful este bine proiectat
            \end{itemize}
            &
            \begin{itemize}
                \item necesită conexiuni suplimentare între clienți
                \item topologie complexă, dificil de administrat
                \item nu există model global inițial furnizat tuturor
                \item performanța depinde de calitatea grafului de conexiuni
                \item nodurile malițioase pot influența direct vecinii
                \item necesită protocoale suplimentare pentru consistența actualizărilor
            \end{itemize}
            \\
            \hline
    
        \end{tabular}
        \caption{Compararea modelelor Federated Learning și Fully Decentralized Learning}
        \label{tab:fl_vs_p2p}
    \end{table}
\end{landscape}
    
In tabelul de mai sus, sunt evidentiate avantajele si dezavantajele utilizarii celor doua tipuri de modele federated learning.

In continuarea acestei lucrari, se va discuta preponderent despre modelul clasic federated learning, fiind unul adoptat la scara larga si care ofera performante bune raportat la costurile de productie.

Modelul clasic la randul sau, se catalogheaza in literatura in functie de tipul dispozitivelor care iau parte la procesul de antrenare. Sub acest filtru, exista Cross-Device Federated Learning si Cross-Silo Federated Learning.

Cross-Device Federated Learning sunt dispozitivele client IOT uzuale, individuale, care comunica orchestratorului printr-un protocol prestabilit.

Cross-Silo Federated Learning sunt dispozitive din institutii guvernamentale, companii, sau centre de date distribuite geografic. Institutiile nu doresc sa schimbe informatii intre ele sau cu un furnizor de servicii central, pastrandu-si confidentialitatea, folosind federated learning pt a antrena propriul model pe datele private ale fiecaruia.

    \subsection{Procesul de antrenare FL}

    Primul pas este stabilirea conexiunii dintre dispozitive si un server de agregare ce permite antrenare distribuita a tipului de retea neuronala sau model ML specific problemei.
Odata stabilit canalul, in faza de configurare initiala, serverul trimite dispozitivelor starea de baza a retelei neuronale, ponderile, in vederea antrenarii individuale.
Fiecare retea se antreneaza cu datele extrase local (on device) si isi imbunatateste configuratia interna la fiecare epoca pentru o perioada de timp bine determinata.

\begin{figure}[h!]
    \centering
    \includegraphics[width=0.7\textwidth]{../images/Device_internal_State.png}
    \caption{Imagine 2.2.3.1: Arhitectura interna a unui dispozitiv}
    \label{fig:nume-label}
\end{figure}

Figura de mai sus descrie operatiile specifice programului Software care se ocupa de antrenarea retelei/modelului. Putem observa cum dispozitivele primesc un plan de antrenare de baza pe care il vor antrena local pe un set de date limitat.

Desi acest pas nu aduce un procent de imbunatatiri foarte mari, in faza urmatoare, dispozitivele vor transmite configuratiile curente ale retelelor lor la orchestrator (server).
Rutina FL\_Runtime extrage configuratia noua locala si ii comunica serverului pentru o posibila actualizare a sa.

In cele din urma, entitatea centrala combina toate aceste ponderi aplicand o functie de agregare si in cazul imbunatatirii setului de ponderi, modifica configuratia de baza si o retransmite dispozitivelor pereche. Daca ponderile noi nu se imbunatatesc semnificativ fata de configuratia de baza, atunci se patreaza aceasta din urma, iar in caz contrar se actualizeaza cu noile ponderi.

\begin{figure}[h!]
    \centering
    \includegraphics[width=0.7\textwidth]{../images/Federated_Learning_Operations.png}
    \caption{Imagine 2.2.3.2: Procedee in invatare automata federata}
    \label{fig:nume-label}
\end{figure}

In figura de mai sus, se pot observa intr-o maniera continua, fluxul de comunicatie dintre dispozitive si serverul agregator, precum si operatiile specifice fiecarei entitati dintr-o runda de mesaje.

Securitatea protocoalelor de agregare, utilizate in comunicatii dintre clienti si orchestrator, este o componenta importanta in procesul federated learning.
De mentionat este faptul ca, in aceasta topologie, comunicatiile au loc criptat, folosind metode specifice precum criptare homomorfica, sau chiar OTP, insa securitatea datelor de pe dispozitive ramane la latitudinea acestuia.

    \subsection{Exemple in viata reala}

    Federated Learning s-a extins rapid în numeroase domenii datorită capacității sale de a antrena modele performante fără a colecta sau centraliza date sensibile.
Prin păstrarea informațiilor la nivelul fiecărui dispozitiv sau instituții, FL reduce riscurile asociate scurgerilor de date și permite colaborarea între entități care altfel nu ar putea împărtăși date brute.
În continuare sunt prezentate câteva exemple reprezentative ale utilizării sale în aplicații din lumea reală.


\subsection*{Industrie și IoT}
\begin{itemize}
    \item \textbf{Mentenanță predictivă:} Vehiculele moderne, utilajele industriale și echipamentele IoT generează constant date despre starea componentelor. FL permite antrenarea unui model comun care poate prezice momentul oportun pentru realizarea mentenanței fără a colecta date brute de la fiecare dispozitiv.
    \item \textbf{Dispozitive de monitorizare:} Senzori purtabili și dispozitive smart home pot furniza statistici privind activitatea sau consumul energetic, păstrând datele utilizatorilor la sursă.
\end{itemize}

\subsection*{Medical}
\begin{itemize}
    \item \textbf{Diagnostic, prognoză și imagistică:} FL este folosit în spitale și clinici pentru detectarea celulelor canceroase din imagini RMN, CT sau radiografii, fără transferul imaginilor către un server central.
    \item \textbf{Confidențialitate menținută la sursă:} Fiecare instituție medicală antrenează local o parte din model, partajând doar actualizările, ceea ce permite colaborarea fără a încălca regulile privind datele pacienților.
\end{itemize}

\subsection*{Financiar}
\begin{itemize}
    \item \textbf{Detectarea fraudelor:} Instituțiile financiare pot îmbunătăți detectarea tranzacțiilor suspecte analizând tipare comune fără a expune date sensibile despre clienți.
\end{itemize}

\subsection*{Servicii și experiență utilizator}
\begin{itemize}
    \item \textbf{Recomandări personalizate:} Platformele de streaming și aplicațiile mobile generează recomandări local, pe dispozitiv, fără a trimite istoricul complet al utilizatorului către server.
    \item \textbf{Analiză comportamentală:} FL poate analiza activitatea utilizatorilor pentru a sugera rutine sănătoase sau îmbunătățiri ale stilului de viață, păstrând confidențialitatea datelor.
\end{itemize}

\subsection*{Securitate și privacy}
\begin{itemize}
    \item \textbf{Supraveghere fără expunerea datelor sensibile:} Modelele de recunoaștere facială pot fi antrenate fără a transmite imagini reale, doar parametrii aferenți.
    \item \textbf{Analiză a sentimentelor:} FL poate analiza reacțiile utilizatorilor la evenimente sociale (like-uri, share-uri, comentarii) fără colectarea directă a acestor date de către platformă.
\end{itemize}


\section{Atacuri de tip Data Poisoning}

In acest capitol se va discuta una dintre avantajele pe care le ofera mediul federated learning atacatorilor si ce inseamna acest lucru pentru fluxul configuratiei modelului. Vom incepe cu definirea
vectorilor de atac si concentrarea pe una dintre categorii, data poisoning. In continuarea lucrarii, vom analiza impactul pe care il are acest tip de atac asupra infrastructurii federate de invatare si riscurile pe care le introduce, precum si cateva propuneri de identificare si constientizare a existentei sale.

    \subsection{Definirea tipurilor de atac}


    In literatura de specialitate, atacurile asupra unui model de ML sau asupra unei retele neuronale sunt definite drept atacuri adversariale (adversarial attacks). Aceasta clasa de atacuri are rolul de a produce modificari in comportamentul normal al modelului intr-un mod indizerabil. Conform lucrarii [1], in functie de nivelul de scop al unui atac, putem avea:

    \begin{itemize}

        \item \textbf{Atacuri fără țintă (untargeted attacks)}  
        Scopul este reducerea acurateții generale a modelului sau chiar destabilizarea completă a acestuia.  
        Un exemplu pentru clasificarea imaginilor este introducerea unui zgomot care degradează calitatea setului de date.  
        O altă metodă este modificarea etichetelor din setul de antrenare, de exemplu atribuirea etichetei \textit{leu} unor imagini care în mod normal ar trebui etichetate drept \textit{pisică}.
    
        \item \textbf{Atacuri țintite (targeted attacks)}  
        Denumite și \textit{backdoor attacks}, deoarece urmăresc modificarea comportamentului modelului doar pentru un anumit subset de date, fără a afecta vizibil acuratețea globală.  
        Continuând exemplul anterior, pentru un anumit tip de imagini ce reprezintă pisici într-o anumită poziție, se poate introduce un artefact vizual menit să păcălească modelul și să clasifice pisica drept \textit{leu}. Astfel apare un backdoor activ doar pentru acele imagini precise.
    
    \end{itemize}
    
    Plecand de la aceasta categorisire, exista o multime de modalitati prin care un atacator poate submina capacitatea de predictie a modelului. Mediul federated learning introduce prin constructie o serie de intrebari la care trebuie gasit un raspuns pentru a determina prin ce moduri un adversar se poate infiltra si poate profita de anumite drepturi pentru a introduce incertitudine in antrenarea sau reglarea (fine-tuning) a modelelor.

    \subsection{Vectori de atac}


    Analiza amenintarilor asupra modelelor de ML, a introdus o serie de posibile vulnerabilitati asupra componentelor ce alcatuiesc fluxul de invatare.
    Un atacator isi poate alege zona de interes, pe baza posibilitatilor de exploatare a sistemului respectiv.

    Vectorii de atac cunoscuti sunt:

    \begin{itemize}
        \item \textbf{Data Poisoning}
        Cand adversarul incearca sa corupa setul de date antrenare cu scopul defectarii modelului inca de la inceput.
        \item \textbf{Model Update Poisoning}
        Cand adversarul se foloseste de o vulnerabilitate ce ii permite modificarea configuratiei parametrilor trimisi catre orchestrator.
        \item \textbf{Evasion attack}
        Cand adversarul are acces la datele de testare si le poate modifica in momentul inferentei.
    \end{itemize}

    In functie de gradul de acces la sistemele gazda, modelul poate fi inspectat in diferite moduri:

    \begin{itemize}
        \item \textbf{Black Box}
        Adversarul nu are abilitatea sa inspecteze parametrii modelului inainte sau in timpul atacului.
        \item \textbf{Stale Box}
        Adversarul poate inspecta doar o versiune incipienta a modelului. Aceasta capabilitate apare si in federated learning cand adversarul are acces la rundele de antrenare ale clientului.
        \item \textbf{White Box}
        Adversarul are abilitatea de a inspecta direct parametrii modelului. Acest scenariu se bazeaza pe un grad de acces superior al adversarului asupra sistemului.
    \end{itemize}

    Pe baza acestor scenarii, atacatorul poate aplica o serie de tehnici pentru modificarea starii modelului. In contextul lucrarii de fata, se va discuta scenariul stale box, adversarul avand
    posibilitatea doar de a introduce un atac Data Poisoning in rundele de antrenare, datele corupte fiind introduse o singura data in procesul de reglare a modelului (fine-tuning).

    \subsection{Atacul Data poisoning}

    Acest tip de atac presupune coruperea datelor de antrenare sau de testare, prin diferite tehnici specifice, la nivelul dispozitivului clientului.
In aceasta paradigma, atacul poate fi considerat la fel de bine targeted sau untargeted intrucat depinde de intentia adversarului si de potentialul risc in divulgarea
punctului de exploatare de care dispune.


În practica atacurilor de tip Data Poisoning, adversarul poate manipula atât conținutul datelor, cât și etichetele acestora, efectele fiind de obicei greu de observat la nivel local.
În mediul federated learning această dificultate este amplificată, deoarece orchestratorul nu are acces direct la datele brute ale clienților. Astfel, orice modificare realizată pe un dispozitiv compromis intră automat în procesul de antrenare, fiind tratată ca o contribuție legitimă.
În mod particular, chiar și un număr redus de exemple otrăvite poate introduce un comportament persistent în modelul global, mai ales dacă atacul este repetat pe durata mai multor runde

In contextul solutiei propuse, se va discuta despre impactul atacului asupra unui set de imagini si tipurile de metode pentru a le altera, asa cum se poate vedea in tabelul 2.3.3.

    \begin{table}[h]
        \centering
        \begin{tabular}{ | p{0.20\linewidth} | p{0.75\linewidth} | }
            \hline
            \textbf{Tip atac} & \textbf{Descriere} \\
            \hline

            \textbf{Gaussian noise}
            &
            Introducerea unui zgomot aleator în imagini sau în vectorii de caracteristici, cu scopul degradării calității datelor și scăderii performanței modelului.
            \\
            \hline

            \textbf{Label flip}
            &
            Modificarea intenționată a etichetelor din setul de antrenare, astfel încât exemple corecte sunt asociate cu clase greșite, afectând procesul de învățare.
            \\
            \hline

            \textbf{Backdoor injection}
            &
            Inserarea unui artefact vizual sau a unui tipar specific într-un subset mic de date, astfel încât modelul să învețe un comportament anormal activat doar de acel trigger.
            \\
            \hline

        \end{tabular}
        \caption{Tipuri de atacuri Data Poisoning}
        \label{tab:data_poisoning_types}
    \end{table}

    Prin aceste tipuri de atacuri adversariale, impactul asupra modelului are loc pe o perioada determinata de timp, de obicei mai lunga, si produce variatii in
predictia finala.

    \subsection{Impactul poisoning in Federated Learning}


Mediul de invatare federata a introdus o serie de amenintari cibernetice preponderent la nivelul dispozitivelor clientilor, acestea fiind cele mai vulnerabile din punctul de vedere
al aplicarii unui atac de otravire a datelor. Clientii sunt producatorii unui model calitativ care sa ofere predictii legitime in diferite scenarii.

Cand se discuta despre alterarea etichetelor (Label Flip Attack) atunci la o prima vedere, utilizatorul nu si-ar da seama decat in urma unei inspectii amanuntite.
Pentru acest tip de atac, exista tehnici de verificare si filtrare ce pot determina daca un tip de informatie este catalogata corect inainte de antrenare.

Efectele unui data poisoning pot persista chiar și după eliminarea datelor corupte, deoarece modelul învață un tipar greșit care nu dispare imediat fără o reantrenare completă.
Acest lucru este relevant mai ales pentru atacurile backdoor, care rămân inactive până la apariția unui trigger vizual, fără a afecta acuratețea generală.
Din acest considerent, atacurile tintite (targeted attacks) sunt deosebit de periculoase in contextul unui mediu de invatare federata pentru ca datele corupte se ascund in
interiorul configuratiilor particulare ale clientilor. Aceste configuratii sunt transmise mai departe la agregator care aplicand functia sa de agregare, amplifica negativ starea modelului.

Având în vedere aceste particularități, devine esențială analizarea metodelor de apărare și a mecanismelor prin care pot fi detectate contribuțiile malițioase.
Provocarile in acest domeniu conduc la o serie tot mai mare de utilitare sau platforme ce permit detectarea facila a acestor atacuri si imbunatatirea evenimentelor cu un
potential risc in organizatii.

Mediul federated learning introduce un risc suplimentar: un adversar care controlează un număr mic de clienți poate influența disproporționat modelul global dacă este integrat într-un moment critic al antrenării.
În absența unor mecanisme robuste de apărare, actualizările malițioase sunt tratate ca fiind legitime, iar agregatorul nu are nicio modalitate directă de a le verifica.

Un alt efect important al acestui tip de atac este degradarea treptată a performanței modelului.
În scenariile untargeted, scăderea acurateții globale poate trece neobservată în primele runde de antrenare, dar devine evidentă odată ce modelul converge către o reprezentare eronată a datelor.
În scenariile targeted, atacul poate compromite decizii critice doar într-un subset de cazuri, ceea ce face detectarea mult mai dificilă și impactul mult mai nociv, mai ales în aplicații sensibile cum ar fi securitatea, domeniul medical sau sistemele autonome.



\section{Alte Notiuni}

In vederea elaborarii solutiei propuse in aceasta lucrare, se vor aminti celelalte concepte care stau la baza implementarii propriu-zise. In acest capitol se vor detalia succint mecanismele ce stau la baza platformei propuse, modul de utilizare si scopul alegerii lor.

    \subsection{Docker}

    Docker reprezintă un set de servicii software de tip platformă ce utilizează virtualizarea la nivel de sistem de operare pentru a crea entități independente, numite containere. Aceste containere sunt create specific pentru a întreprinde anumite acțiuni și oferă un mediu izolat de execuție.

    Un container este o instanta software ce vine impachetata cu programul aplicatiei si toate bibliotecile necesare dezvoltarii ei.
    O imagine este vizualizata drept un sablon de instructiuni pentru crearea unui container cu un anumit tip de biblioteci necesare dezvoltarii unei aplicatii.
    De obicei, imaginile sunt construite pe baza unui fisier denumit Dockerfile care propune o serie de comenzi pentru crearea unui mediu personalizat.
    
    Toate containerle Docker ruleaza prin intermediul Docker Engine, un serviciu ce ruleaza la nivelul sistemului de operare si ofera suport cross-platform (Linux, Windows, macOS).
    
    Containerizarea diferă de virtualizarea tradițională prin faptul că containerele partajează același kernel al sistemului de operare gazdă, în timp ce mașinile virtuale (VM) necesită fiecare un sistem de operare complet. Această abordare face containerele Docker mult mai ușoare (de ordinul MB vs GB) și mai rapide la pornire (secunde vs minute) comparativ cu VM-urile.
    
    Arhitectura platformei docker este asemanatoare modelului client-server, fiind compusa din urmatoarele componente:
    \begin{itemize}
        \item \textbf{Dockerd}
        un proces daemon, identificabil drept server, ce gestioneaza tot fluxul de servicii, de la imagini si containere pana la volume si retele.
    
        \item \textbf{Aplication Binary Interfaces (API)}
        o suita de interfete de comunicare si control al serverului
    
        \item \textbf{Comanda docker}
        o interfata in linie de comanda, docker
    \end{itemize}
    
    În contextul platformei dezvoltate, Docker ofera un mediu de dezvoltare automat in care se regaseste gazduita platforma publica de simulare. Toate informatiile despre fiecare simulare sunt stocate la nivelul unei baze de date PostgreSQL, in timp ce aplicatia web este publica printr-un container frontend, iar in spate regasim un container backend pentru transmiterea de comenzi serverului.
    
    Comunicatia dintre containere are loc in aceeasi retea locala docker, iar informatiile despre fiecare simulare a clientului persista in acelasi volum partajat.
    
    Pe langa Dockerfile, serviciul Docker Engine ofera si posibilitatea crearii unei configuratii prestabilite pentru definirea de topologii de retea de containere prin Docker Compose. Pe baza fisierelor Dockerfile in care sunt definite sabloanele imaginilor si a unui fisier de configurare YAML a topologiei relatiilor dintre containere, serviciul Docker Compose permite implementarea unei infrastructuri intregi prin rularea si stergerea sa dintr-o serie de comenzi.
    
    \subsection{Python}

    \subsection{Rest API}



\chapter{Proiectare, Implementare si Testare}

    \section{Cerintele Software}

        \subsection{Cerintele functionale}
        \subsection{Cerintele nefunctionale}
    
    \section{Arhitectura platformei}

        \subsection{Containere}
        \subsection{Server}
    
    \section{Testare}

\chapter{Rezultate si Metrici Simulari}

    \section{Evaluare Performante}

        \subsection{Scalabilitatea Simularilor}
        \subsection{Scalabilitatea platformei}

    \section{Evaluare Rezultate}

        \subsection{Performante Gaussian Noise}
        \subsection{Performante Label-Flip}
        \subsection{Performante Backdoor}

\chapter{Concluzii si dezvoltare ulterioara}

    \section{Starea Curenta}
    \section{Dezvoltare Ulterioara}


\section{Tabele}

Tabelele sunt aranjări a informației într-o structură formată din linii și coloane, care permite o mai bună observare a acesteia.

Mai jos apar două exemple. Primul tabel este de dimensiune mică. Al doilea, din cauza dimensiunii mai mari, are o orientare inversată și este plasat singur pe o pagină.

\input{components/tables/small_table}

\newpage

\begin{landscape}
    \vspace*{\fill}
    \input{components/tables/large_table}
    \vspace*{\fill}
\end{landscape}

\newpage

\section{Imagini}

Imaginile sunt utilizate în cadrul lucrării pentru exemplificarea unor idei în manieră vizuală.

\begin{center}
    \includegraphics[width=7cm]{components/images/architecture.jpg}
    \label{fig:architecture}
    \captionsetup{justification=centering,margin=1cm}
    \captionof{figure}[Arhitectura unui calculator]{Arhitectura unui calculator\footnotemark}
\end{center}
\vspace{0.3cm}

\footnotetext{Arhitectura ilustrată este de fapt cea von Neumann.}

\section{Liste}

Listele sunt simple serii de informații.

\begin{itemize}
    \item Un item
    \item Unul dintre itemi
    \item Încă un item
\end{itemize}

Acestea pot conțin itemi identificați prin numere dacă indexarea sau sortarea sunt necesare.

\begin{enumerate}
    \item Primul item
    \item Al doilea item
    \item Al treilea item
\end{enumerate}

\section{Formule Matematice}

\LaTeX{} oferă un mod programatic de a construi formule matematice, după cum este cea de mai jos.

\newpage

$ \sum \mathbf{F} = 0 \Leftrightarrow {\frac {\mathrm{d} \mathbf{v}}{\mathrm{d} t}} = 0 $

\section{Note de Subsol. Citări}

Notele de subsol pot fi utile în cazul explicațiilor suplimentare (cum a fost cea referitoare la imaginea inclusă, la care sintaxa este puțin diferită din cauza plasării notei în cadrul legendei) sau a citărilor\footnote{\fullcite{morphological_operations}} care nu se pretează a fi trecute în bibliografie din cauza utilizării lor punctuale.

Pe de altă parte, sursele bibliografice citate intens \cite{cloud_crypto} sunt marcate corespunzător și notate în bibliografie.

\section{Etichete. Referințe}

În cadrul surselor \LaTeX{} a acestui document, apar \textit{tag}-uri \inlinecode{\textbackslash label} care creează o etichetă utilă referințelor interne. Acestea din urmă indică elemente din cadrul documentului curent (de exemplu, către tabelul \ref{tab:small_table}).

Mai pot apărea referințe externe, către resurse din Internet (de exemplu, către \textit{website}-ul \href{https://www.wikipedia.org/}{Wikipedia}).

\end{document}